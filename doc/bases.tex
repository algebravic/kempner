\documentclass{article}

\usepackage{amsmath}
\usepackage{amsfonts}
\usepackage{amsthm}
\newtheorem{lemma}[section]{Lemma}
\newtheorem{theorem}[section]{Theorem}
\newtheorem{corollary}[section]{Corollary}
\newtheorem{proposition}[section]{Proposition}
\newtheorem{definition}[section]{Definition}
\newtheorem{notation}[section]{Notation}
\newcommand{\lemref}[1]{lemma~\ref{lem:#1}}
\newcommand{\corref}[1]{lemma~\ref{cor:#1}}
\newcommand{\propref}[1]{lemma~\ref{prop:#1}}
\newcommand{\defref}[1]{lemma~\ref{def:#1}}
\newcommand{\CC}{\mathbb{C}}
\newcommand{\RR}{\mathbb{R}}
\newcommand{\ZZ}{\mathbb{Z}}
\newcommand{\NN}{\mathbb{N}}
\title{A Kempner like series with primes}
\author{Victor S. Miller}
\begin{document}
\maketitle
\begin{abstract}
  Neil Sloane posed the following problem: For a nonnegative integer
  $n$ let $V(n)$ be the function obtained by using the base 10 digits
  of $n$ as base 11 digits. It is straightforward to see that the
  series $\sum_{n=1}^\infty \frac{1}{V(n)}$ converges. Thus the series
  $\sum_{p \text{ prime}} \frac{1}{V(p)}$ also converges. The problem
  is to obtain a good approximation to its value. In order to do this
  we construct a sequence of increasingly good approximations to its
  value obtained as certain linear combination of values of the
  \emph{prime zeta function}
  $Z(s) := \sum_{p \text{ prime}} \frac{1}{p^s}$.
\end{abstract}

\section{The Problem}
\label{sec:problem}

It is well known that the \emph{harmonic series}
\begin{displaymath}
  \sum_{n=1}^\infty \frac{1}{n}
\end{displaymath}
diverges. In fact, we have
\begin{displaymath}
  \sum_{n=1}^n \frac{1}{n} \sim \log N
\end{displaymath}
as $n \rightarrow \infty$.

Kempner in \cite{kempner1914curious}
considered series of the form
\begin{displaymath}
  \sum_{n \in C} \frac{1}{n},
\end{displaymath}
where $C$ is a set of nonnegative integers whose base $b$
representation does not use a particular ``digit'' $0 \le c <
b$. Kempner pointed out that such series converge.

Neil Sloane asked about evaluating the following two series
\begin{equation}
  \label{eq:sloane1}
  \sum_{n=1}^\infty \frac{1}{V(n)}
\end{equation}
and
\begin{equation}
  \label{eq:sloane2}
  \sum_{p \text{ prime}} \frac{1}{V(p)},
\end{equation}
where $V(n)$ is the function from nonnegative integers to
nonnegative integers which uses the base 10 digits of $n$ as base 11
digits. It's easy to see that \eqref{eq:sloane1} is one of Kempner's
series.  This follows since we may interpret the set
$\{V(n) : n \in \NN\}$ as the set of integers whose base 11
representation does not use the ``digit'' 10.  Since
\eqref{eq:sloane2} is a subseries of \eqref{eq:sloane1}, it too
converges.

In \cite{baillie2008summing} the author describes an efficient
algorithm to find good approximations to the sums of a Kempner series.

One way to specify this is to require that
\begin{equation}
  \label{eq:basic}
  V(10 n + a) = 11 V(n) + a,
\end{equation}
for all integers $n \ge 0, 0 \le a < 10$. Note that $V(a) = a$ for all
$0 \le a < 10$ follows immediately. Note further that if $0 \le a <
10^b$ for some $b > 0$ then $V(a) < 11^b$.


A particular case of a ``Kempner'' series is
\begin{displaymath}
  \sum_{n=1}^\infty \frac{1}{V(n)}.
\end{displaymath}
 
In this paper we consider the series
\begin{equation}
  \label{eq:main:series}
  \sum_{p \text{ prime}} \frac{1}{V(p)}.
\end{equation}
Since the set of terms of this series is a subset of the terms of
\eqref{eq:basic} it, too, converges. Our goal is to find a good
approximation for its value. Direct summation of this series converges
excruciatingly slowly. Instead we express the value in terms of values
of the \emph{prime zeta function}.

In order to find a good approximation to the value of this series we
proceed as follows:

\begin{enumerate}
\item We find good upper and lower bounds for $V(p)$ which only depend
  on $p$ and the high order digits of $p$.
\item We write the series \eqref{eq:main:series} as a sum of a finite
  set of sub-series
  each of which depends on the high order decimal digits of the prime
  $p$.
\item We evaluate each such series by using Fourier analysis to get
  good smooth upper and lower approximations for the characteristic
  functions of an interval, and use the Fourier coefficients of these
  functions as coefficients of values of the prime zeta function.
\end{enumerate}
\section{Decomposition}
\label{sec:decomp}

\begin{theorem}
  \label{thm:main}
  Let $b$ be a positive integer. For each integer $10^b \le a <
  10^{b+1}$ there are complex numbers $c_{a, j}$ such that
  \begin{equation}
    \label{eq:decomp}
    \sum_{p \text{ prime}} \frac{1}{V(p)} = \sum_{p < 10^b, \text{
        prime}}\frac{1}{V(p)} + \sum_{10^b \le a < 10^{b+1}}
    \sum_{j=1}^\infty \Re c_{a,j}Z_b\left(\frac{\log 11}{\log 10} + 2 \pi i a j\right),
  \end{equation}
  where $Z_b(s) := \sum_{p \ge 10^b \text{ prime}} \frac{1}{p^s}$ is
  the truncated \emph{prime zeta function}. Morevor the coefficients
  $c_{a,j}$ decay super polynomially with $j$.
\end{theorem}
\begin{lemma}
  \label{lem:recursion}
  Let $n \ge 0, b > 0, 0 \le c < 10^b$ be integers. Then we have
  \begin{equation}
    \label{eq:recur}
    V(10^b n + c) = V(n) 11^b + V(c).
  \end{equation}
\end{lemma}
\begin{proof}
  We use induction on $b$. The base case is in the definition of $V$.
  We then have, for $0 \le c < 10^{b+1}$, $c = 10^b c' + c''$ with $0
  \le c' < 10, 0 \le c'' < 10^b$
  \begin{displaymath}
    \begin{aligned}[t]
      V(10^{b+1}n + c) & = V(10^b (10n + c') + c'') \\
      & = V(10n + c') 10^b + V(c'') \\
      & = (11 V(n) + c') 11^b + V(c'') \\
      & = 11^{b+1} V(n) + V(c),
    \end{aligned}
  \end{displaymath}
  The second and fourth equalities are by induction, and the third
  equality is by definition of $V$.
\end{proof}
\begin{definition}
  If $n \ge 0, a > 0$ we say that the base 10 expansion of $n$
  \emph{starts with $a$} if, for some integer $b \ge 0$, we have
  \begin{equation}
    \label{eq:bracket}
    a 10^b \le n \le a 10^b + 10^b - 1.
  \end{equation}
\end{definition}
\begin{notation}
  \label{not:starts}
  Given an integer $a \ge 1$ define the function $\chi_a: \NN
  \rightarrow \{0,1\}$ by $\chi_a(n) = 1$ if $n$ starts with $a$ and 0
  otherwise.
\end{notation}
\begin{lemma}
  \label{lem:covering}
  Every integer $n \ge 10^b$ starts with some element of the set
  $S = \{a \in \ZZ: 10^b \le a < 10^{b+1}\}$.
\end{lemma}
\begin{proof}
  Let $d = \lfloor \frac{\log n }{\log 10}\rfloor$,
  then $d \ge b$ and $10^d \le n < 10^{d+1}$. Let
  $a = \lfloor \frac{n}{10^{d-b}} \rfloor$. Then
  $10^b \le a < 10^{b+1}$ and
  $a 10^{d-b} \le n < (a+1) 10^{d-b}$. Thus $n$ starts with $a \in S$.
\end{proof}
\begin{proposition}
  \label{prop:expansion}
  Let $b \ge 0$ be an integer. Then
  \begin{equation}
    \label{eq:decomposition}
    \sum_{p \text{ prime}} \frac{1}{V(p)} =
    \sum_{p < 10^b \text{ prime}} \frac{1}{V(p)} +
    \sum_{10^b \le a < 10^{b+1}} \sum_{p > 10^b \text{ prime}} \frac{\chi_a(p)}{V(p)}.
  \end{equation}
\end{proposition}
\begin{proposition}
  \label{prop:bounds}
  Let $a > 0, n > 0$ be integers. If the base 10 expansion of
  $n$ starts with $a$
  then
  \begin{equation}
    \label{eq:leading:digits}
    \left( \frac{V(a)}{(a+1)^{\log 11/\log 10}}\right)
      n^{\frac{\log 11}{\log 10}}
      < V(n) < 
      \left( \frac{V(a) + 9/10}{a^{\log 11/\log 10}}\right)
      n^{\frac{\log 11}{\log 10}}.
  \end{equation}
\end{proposition}
\begin{proof}
  If there is a $b \ge 0 $ such that \eqref{eq:bracket} holds, there is an integer $0 \le c < 10^b$
  such that $n = a 10^b + c$. Thus $V(n) = V(a) 11^b + V(c)$ by
  \lemref{recursion}. However $0 \le V(c) \le \sum_{j=0}^{b-1} 9 \cdot
  11^j < (9/10) 11^b$.
  Thus
  \begin{equation}
    \label{eq:vbracket}
    V(a) 11^b \le V(n) < (V(a) + 9/10)11^b.
  \end{equation}
  The equation \eqref{eq:leading:digits} follows from the inequality
  \eqref{eq:bracket} by taking logarithms:
  \begin{displaymath}
    \frac{\log n}{\log 10} - \frac{\log(a+1)}{\log 10} < b
    \le \frac{\log n}{\log 10} - \frac{\log a}{\log 10},
  \end{displaymath}
  and substituting into \eqref{eq:vbracket}.

\end{proof}
If we take $b=0$ and set $\gamma = \min_{a=1}^9
\frac{V(a)}{(a+1)^{\frac{\log 11}{\log 10}}} > 0$, then $V(n) > \gamma n^{\log 11/\log 10}$.
Thus the series $\sum_n \frac{1}{V(n)}$ is bounded
term by term by the series $\sum_n \frac{1}{\gamma n^{\log 11/\log
    10}}$ which is absolutely convergent.
\begin{definition}
  The \emph{prime zeta function}
  \begin{displaymath}
    Z(s) := \sum_{p \text{ prime}} \frac{1}{p^s},
  \end{displaymath}
  where $s \in \CC, \Re s > 1$.
\end{definition}
\begin{lemma}
  \label{lem:interval}
  Let $a$ be a positive integer, $b = \lfloor \log a/\log 10\rfloor$,
  and $F(x) := x - \lfloor x \rfloor$, the fractional part of $x$.
  Define the interval $I_a = [F(\log a/\log 10), F(\log (a+1)/\log
  10)$ if $a < 10^b-1$. If $a=10^b-1$, then $I_a = [F(\log a/\log
  10), 1)$.
  A positive integer $n\ge 10^b$ starts with $a$ if and only if
  $F(\log n\log 10) \in I_a$.
\end{lemma}
\begin{proof}
  By definition, $n$ starts with $a$ if $a 10^c \le n < (a+1)10^c$,
  for some integer $c$. By definition,
  $F(\log a/\log 10) = \log a/\log 10 - b$, where $b = \lfloor \log
  a/\log 10 \rfloor$. Thus
  \begin{displaymath}
    \log a/\log 10 -b \le \log n/\log 10 - c -b < \log(a+1)/\log 10 - b.
  \end{displaymath}
\end{proof}
\begin{lemma}
  \label{lem:upper:bound}
  Let $\Phi(x)$ be a integrable function supported in $[-1,1]$ such
  that $\Phi(x) \ge 0$, and $\int_{-1}^1 \Phi(x) = 1$.  For
  $\frac{1}{4} \ge \varepsilon > 0$, set
  $\Phi_\varepsilon(x) := \varepsilon^{-1} \Phi \left(
    \frac{x}{\varepsilon} \right)$.  Given $a < b$, and $\delta$ Let
  $f_\delta(x)=1$ if $a + \delta \le x \le b - \delta$, and and 0
  otherwise. Then
  $0 \le \Phi_\varepsilon \star f_\delta(x) \le f_0(x)$ for all $x$,
  if $\delta \ge 0$ and
  $\Phi_\varepsilon \star f_\delta(x) \ge f_0(x)$is $\ge f_0(x)$ if $\delta < 0$.
\end{lemma}
\begin{proof}
  Define $\Psi(x) = \int_{\max(-1, x)}^{\min(1,x)} \Phi(t) dt$. Then
  $\Psi(x) = 0$ if $x \le -1$, $\Psi(x) = 1$ if $x \ge 1$ and $\Psi(x)$
  is non-decreasing.
  We have
  \begin{displaymath}
    \begin{aligned}
      \Phi_\varepsilon \star f_\delta(x) & = \varepsilon^{-1}\int_{a +
      \delta}^{b - \delta}
      \Phi((x - t)/\varepsilon) dt \\
      & = \Psi((x-a - \delta)/\varepsilon) -
      \Psi((x-b+\delta)/\varepsilon) \le 1.
    \end{aligned}
  \end{displaymath}
If $(x-b + \delta)/\varepsilon \ge 1$ or $(x-a-\delta)/\varepsilon \le
-1$ the right hand side is 0. Those conditions are $x \ge b + \varepsilon
- \delta$ or $x \le a - \varepsilon + \delta$. If $\delta = \epsilon$
this then shows that $0 \le \Phi_\varepsilon \star f_\varepsilon(x) \le f_0(x)$.
On the other hand, if $x \ge a + \varepsilon + \delta$, and $x \le b -
\delta - \varepsilon$ then the right hand side is 1.
If $\delta = - \varepsilon$, this shows that $\Phi_\varepsilon \star
f_{-\varepsilon}(x) \ge f_0(x)$.
On the other hand, we have $||f_\varepsilon - f_{-\varepsilon}||_1 = 4
\varepsilon$. Thus $||\Phi_\varepsilon \star f_\varepsilon - 
\Phi_\varepsilon \star f_{-\varepsilon}||_\infty \le 4 \varepsilon
||\Phi_\varepsilon||_\infty \le 4 \varepsilon$.
\end{proof}
If we pick a function $\Phi$ as above which is $\mathcal{C}^\infty$
and has all derivatives at $1$ and $-1$ equal to 0, then convolving it
with the indicator function of an interval will produce a ``nice''
function 
\nocite{johnson2015saddle}
\bibliography{series}
\bibliographystyle{plain}
\end{document}
