\documentclass{article}

\usepackage{amsmath}
\usepackage{amsfonts}
\usepackage{amsthm}
\usepackage{hyperref}
\newtheorem{lemma}[section]{Lemma}
\newtheorem{theorem}[section]{Theorem}
\newtheorem{corollary}[section]{Corollary}
\newtheorem{proposition}[section]{Proposition}
\newtheorem{definition}[section]{Definition}
\newtheorem{notation}[section]{Notation}
\newcommand{\lemref}[1]{lemma~\ref{lem:#1}}
\newcommand{\corref}[1]{lemma~\ref{cor:#1}}
\newcommand{\propref}[1]{lemma~\ref{prop:#1}}
\newcommand{\defref}[1]{lemma~\ref{def:#1}}
\newcommand{\CC}{\mathbb{C}}
\newcommand{\RR}{\mathbb{R}}
\newcommand{\ZZ}{\mathbb{Z}}
\newcommand{\NN}{\mathbb{N}}
\title{A Tale of two Series}
\author{Victor S. Miller}
\begin{document}
\maketitle
\begin{abstract}
  Neil Sloane posed the following problem: For a nonnegative integer
  $n$ let $V(n)$ be the function obtained by using the base 10 digits
  of $n$ as base 11 digits. It is straightforward to see that the
  series $\sum_{n=1}^\infty \frac{1}{V(n)}$ converges. Thus the series
  $\sum_{p \text{ prime}} \frac{1}{V(p)}$ also converges. The problem
  is to obtain a good approximation to its value. In order to do this
  we construct a sequence of increasingly good approximations to its
  value obtained as certain linear combinations of values of the
  \emph{prime zeta function}
  $P(s) := \sum_{p \text{ prime}} \frac{1}{p^s}$.
\end{abstract}

\section{The Problem}
\label{sec:problem}

When confronted with an infinite series
$$\sum_{i=1}^\infty a_n$$
two questions arise: First, does the series converge? And second, if
it does converge, find a good approximation to its value. If a series
is proved to converge, then we have the following strategies for
approximating its value:

\begin{enumerate}
\item \emph{Brute Force}: Choose a large $N$ and explictly sum (up to a
  certain accuracy) $\sum_{i=1}^N a_n$.
\item \emph{Smooth Approximation}: If there is smooth function $f(x)$
  such that $f(n) = a_n$ for all sufficiently large $n$, then we can
  use methods like
  the \emph{Euler-MacLaurin} formula.
\item \emph{Transformation}: By using the structure of the terms $a_n$
  we create another series $\sum_n b_n$ where the $b_n$ are derived
  from the $a_n$.
\end{enumerate}

It is well known that the \emph{harmonic series}
\begin{displaymath}
  \sum_{n=1}^\infty \frac{1}{n}
\end{displaymath}
diverges. In fact, we have
\begin{displaymath}
  \sum_{n=1}^n \frac{1}{n} \sim \log N
\end{displaymath}
as $n \rightarrow \infty$.

Kempner in \cite{kempner1914curious}
considered series of the form
\begin{displaymath}
  \sum_{n \in C} \frac{1}{n},
\end{displaymath}
where $C$ is a set of nonnegative integers whose base 10\footnote{His
  argument is easily generalized to any integral base}
representation does not use a particular ``digit'' $0 \le c <
10$. Kempner pointed out that this restricted sum converges.

\begin{notation}
  Let $A \subseteq \NN$ be a subset of the positive integers. For $x >
  0$ denote by $A(x) = \#\{n \in A: n \le x\}$ denote the number of
  elements of $A$ that are $\le x$.
\end{notation}
\begin{proposition}[Abel's Partial Summation formula]
\label{prop:partial}
  Let $a_i, b_i \in \RR$ for $i=1, \dots, N$, and $A_j = \sum_{j=1}^i a_i$.
  \begin{displaymath}
    \sum_{i=1}^N a_i b_i = \sum_{i=1}^{N-1} A_i (b_i - b_{i+1}) +
    A_{N-1}b_N.
  \end{displaymath}
\end{proposition}
\begin{corollary}
  Let $A \subseteq \NN$. If $A(x) = O(x^\alpha)$ for $0 \le \alpha <
  1$, then $\sum_{n \in A}\frac{1}{n}$ converges.
\end{corollary}
\begin{proof}
  Apply \propref{partial} with $a_n = 1$ if $n \in A$ and 0 otherwise,
  and $b_n = \frac{1}{n}$. We then have
  \begin{displaymath}
    \sum_{n \in A, n \le N}\frac{1}{n} = \sum_{n=1}^{N-1}
    \frac{A(n)}{n(n+1)} + \frac{A(N)}{N}.
  \end{displaymath}
  By hypothesis, we have $\frac{A(n)}{n(n+1)} = O(n^{- (2 - \alpha)})$,
  and $0 \le \sum_{n=1}^{N-1} n^{- (2 - \alpha)} \le \int_1^N
  \frac{dx}{x^{2-\alpha}}$ which converges.
\end{proof}

Namely,
for $i \ge 1$ let $C_i := \{n \in C: 10^{i-1} \le n < 10^i\}$. Since
$C$ omits one digit, $C_i$ has at most $9^i$ elements. If we set
$s(i) = \sum_{n \in C_i} \frac{1}{n}$. We have
$\frac{9^i}{10^i} \le s(i) \le \frac{9^i}{10^{i-1}}$. Thus
\begin{displaymath}
  9 \le \sum_{n \in C} \frac{1}{n} = \sum_{i=1}^\infty s(i) \le 90.
\end{displaymath}

Baillie \cite{baillie1979sums} and Burnol \cite{burnol2024moments}
have given efficient algorithms to accurately sum such series.
Since Kempner's original paper, and a followup paper by Irwin 
\cite{irwin1916curious} discussed other harmonic series with other
restrictions on the base 10 digits in which there are bounds on the
number of occurences of a specific digit.

If $S$ is a set of positive integers, and $x$ is real, denote by $S(x)
:= \#\{n \in S: n \le x\}$. We call a set $S$ \emph{thin} if
$\lim_{x \rightarrow \infty} \frac{S(x)}{x} = 0$. We call a set $S$
\emph{harmonically thin} if $\sum_{n \in S} \frac{1}{n}$ converges.
It's fairly straightforward to see that if $S$ is harmonically thin,
then it is thin, but the converse does hold.

Denote by $S_h(x) = \sum_{n \in S, n \le x}\frac{1}{n}$, and $H_x =
\sum_{1 \le n \le x} \frac{1}{n}$.

We have the Stieltjes integral
\begin{displaymath}
  \sum_{n \in S, n \le y} \frac{1}{n} = \int_{\frac{1}{2}}^y \frac{1}{x} d S(x) =
\left.\frac{S(x)}{x} \right|_{\frac{1}{2}}^y + \int_{\frac{1}{2}}^y \frac{S(x)}{x^2} dx.
\end{displaymath}

Also
\begin{displaymath}
  \frac{S(y)}{y} = \int_{\frac{1}{2}}^y x d S_h(x) = \left. \frac{x S_h(x)}{y}
  \right|_{\frac{1}{2}}^y - \frac{1}{y} \int_{\frac{1}{2}}^y S_h(x) dx.
\end{displaymath}

However, the set $\mathcal{P}$ of primes is harmonically
thick. Indeed, we have the theorem of Mertens:

\begin{equation}
\label{eq:mertens}
\sum_{p \in \mathcal{P}, p \le x} \frac{1}{p} = \log \log x + c +
O\left(\frac{1}{x}\right)
\end{equation}
for some constant $c > 0$\footnote{There is a quip ``$\log \log x$
  approaches infinity, but has never been observed to do so.''}

Once we now that a sum of a series converges, a natural question is:
``What is its value?''.

We'd like to understand those sets, $S$, of positive integers, such
that
\begin{equation}
  \label{eq:thin}
  \sum_{n \in S} \frac{1}{n}
\end{equation}
converges. We will call such sets \emph{harmonically thin}. All other
sets of integers we will call \emph{harmonically thick}. 

Neil Sloane asked about evaluating the following two series
\begin{equation}
  \label{eq:sloane1}
  \sum_{n=1}^\infty \frac{1}{V(n)}
\end{equation}
and
\begin{equation}
  \label{eq:sloane2}
  \sum_{p \text{ prime}} \frac{1}{V(p)},
\end{equation}
where $V(n)$ is the function from nonnegative integers to
nonnegative integers which uses the base 10 digits of $n$ as base 11
digits. It's easy to see that \eqref{eq:sloane1} is one of Kempner's
series.  This follows since we may interpret the set
$\{V(n) : n \in \NN\}$ as the set of integers whose base 11
representation does not use the ``digit'' 10.  Since
\eqref{eq:sloane2} is a subseries of \eqref{eq:sloane1}, it too
converges.

In \cite{baillie1979sums} the author describes an efficient
algorithm to find good approximations to the sums of a Kempner series.

One way to specify this is to require that
\begin{equation}
  \label{eq:basic}
  V(10 n + a) = 11 V(n) + a,
\end{equation}
for all integers $n \ge 0, 0 \le a < 10$. Note that $V(a) = a$ for all
$0 \le a < 10$ follows immediately. Note further that if $0 \le a <
10^b$ for some $b > 0$ then $V(a) < 11^b$.

\section{Numeration Systems and Finite Automata}
\label{sec:numeration}

In this section we discuss the methods of Baillie and Burnol of
accelerating ellipsephic harmonic series, and generalize them.

In \cite{kempner1914curious} Kempner considered the following series:
\begin{equation}
  \label{eq:kempner}
  \sum_{n \in C}\frac{1}{n},
\end{equation}
where $C$ is the set of positive integers whose base 10 representation
is missing a fixed, digits, $a$.
He observed, that since the number
of $i$-digit numbers that are missing a digit is $\le 9^i$, that the
contribution to the sum \eqref{eq:kempner} of the terms with exactly
$i$ digits is $\le \frac{9}{10}^i$. Since the geometric series
$\sum_i \frac{9}{10}^i$ converges so does the series in question (by
the comparison test). In particular the sum is then $\le 90$.

Baillie, in \cite{baillie1979sums}, gave an efficient algorithm which
will evaluate such sums with high accuracy, which we review and generalize.

Let $b > 1$ be an integer, and $0 \le a < b$ be a ``digit''. Define
the set $C_a$ to be the set of positive integers whose base $b$
representation does not use the ``digit'' $a$.

\begin{definition}
  If $b > 1$ is an integer, a \emph{$b$-word} is a finite sequence
  $w = (w_1, \dots, w_k)$ of integers $0 \le w_i < b$ and $w_k \ne 0$.

  The \emph{value} $\nu(w) := \sum_{i=1}^k w_i b^{i-1}$. The \emph{length}
  $\ell(w) := k$.
  
  A set of $b$-words is called a \emph{language}. If $L$ is a
  language, we denote by $L_i$ the set of words in $L$ of length $i$.

\end{definition}

Note that the function $\nu: L \rightarrow \NN$ is injective, and if
$\ell(w) < \ell(w')$ then $\nu(w) < \nu(w')$. If $x>0$ denote by $L(x)
= \#\{w \in L : \nu(w) \le x\}$. For $i > 0$, an integer,  denote by
$L^{(i)} = \{ w \in L : \ell(w) = i\}$, the set of words in $L$ of
length $i$.

Given such a language we are interested in the series
\begin{displaymath}
  \sum_{w \in L} \frac{1}{\nu(w)}.
\end{displaymath}
We observe that if $L(x) = O(x^\alpha)$ for $0 < \alpha < 1$, then
this series converges. Since the function $L(x)$ has bounded variation
the value of the the sum may expressed as a Stieltjes integral.
\begin{equation}
  \label{eq:stieltjes}
  \sum_{w \in L, \mu(w) \le x} = \int_{\frac{1}{2}}^x\frac{d L(t)}{t}.
\end{equation}
Using integration by parts, this is
\begin{equation}
  \label{eq:parts}
  \left. \frac{L(t)}{t} \right|_{\frac{1}{2}}^x + \int_{\frac{1}{2}}^x
  \frac{L(t)}{t^2} dt.
\end{equation}
Kempner's original series used $b=10$ where the restriction was
imposed that for all $w \in L$, we have $w_i \ne a$ for all $i=1,
\dots, \ell(w)$.

Kempner then observed that, for his language, we have $L(x) =
O(x^{\frac{\log 9}{\log 10}})$. Namely, $|L^{(i)}| \le 9^i$, since
there at most 9 choices for each digit. We then have $L(10^k) \le
sum_{i=1}^k 9^i \le \frac{9}{8} 9^k$

A particular case of a ``Kempner'' series is
\begin{displaymath}
  \sum_{n=1}^\infty \frac{1}{V(n)}.
\end{displaymath}
 
In this paper we consider the series
\begin{equation}
  \label{eq:main:series}
  \sum_{p \text{ prime}} \frac{1}{V(p)}.
\end{equation}
Since the set of terms of this series is a subset of the terms of
\eqref{eq:basic} it, too, converges. Our goal is to find a good
approximation for its value. Direct summation of this series converges
excruciatingly slowly. Instead we express the value in terms of values
of the \emph{prime zeta function}.

In order to find a good approximation to the value of this series we
proceed as follows:

\begin{enumerate}
\item We find good upper and lower bounds for $V(p)$ which only depend
  on $p$ and the high order digits of $p$.
\item We write the series \eqref{eq:main:series} as a sum of a finite
  set of sub-series
  each of which depends on the high order decimal digits of the prime
  $p$.
\item We evaluate each such series by using Fourier analysis to get
  good smooth upper and lower approximations for the characteristic
  functions of an interval, and use the Fourier coefficients of these
  functions as coefficients of values of the prime zeta function.
\end{enumerate}
\section{Decomposition}
\label{sec:decomp}

\begin{theorem}
  \label{thm:main}
  Let $b$ be a positive integer. For each integer $10^b \le a <
  10^{b+1}$ there are complex numbers $f_{a, j}$ such that
  \begin{equation}
    \label{eq:decomp}
    \sum_{p \text{ prime}} \frac{1}{V(p)} = \sum_{p < 10^c, \text{
        prime}}\frac{1}{V(p)} + \sum_{a \in S_c}
    \sum_{j=1}^\infty \Re f_{a,j}P_c\left(\frac{\log 11}{\log 10}
      - \frac{2 i \pi a j}{\log 10}\right),
  \end{equation}
  where $P_c(s) := \sum_{p \ge 10^c \text{ prime}} \frac{1}{p^s}$ is
  the truncated \emph{prime zeta function}. Morevor the coefficients
  $f_{a,j}$ decay super polynomially with $j$.
\end{theorem}
\begin{lemma}
  \label{lem:recursion}
  Let $n \ge 0, b > 0, 0 \le c < 10^b$ be integers. Then we have
  \begin{equation}
    \label{eq:recur}
    V(10^b n + c) = V(n) 11^b + V(c).
  \end{equation}
\end{lemma}
\begin{proof}
  We use induction on $b$. The base case is in the definition of $V$.
  We then have, for $0 \le c < 10^{b+1}$, $c = 10^b c' + c''$ with $0
  \le c' < 10, 0 \le c'' < 10^b$
  \begin{displaymath}
    \begin{aligned}[t]
      V(10^{b+1}n + c) & = V(10^b (10n + c') + c'') \\
      & = V(10n + c') 10^b + V(c'') \\
      & = (11 V(n) + c') 11^b + V(c'') \\
      & = 11^{b+1} V(n) + V(c),
    \end{aligned}
  \end{displaymath}
  The second and fourth equalities are by induction, and the third
  equality is by definition of $V$.
\end{proof}
\begin{definition}
  If $n \ge 0, a > 0$ we say that the base 10 expansion of $n$
  \emph{starts with $a$} if, for some integer $c \ge 0$, we have
  \begin{equation}
    \label{eq:bracket}
    a 10^c \le n \le a 10^c + 10^c - 1.
  \end{equation}
  This is equivalent to
  \begin{equation}
    \label{eq:starts}
    \log a \le \log n - c \log 10 < \log (a+1)
  \end{equation}
  for some integer $c$.
\end{definition}
\begin{notation}
  \label{not:starts}
  Given an integer $a \ge 1$ define the function $\chi_a: \NN
  \rightarrow \{0,1\}$ by $\chi_a(n) = 1$ if $n$ starts with $a$ and 0
  otherwise.
\end{notation}
\begin{lemma}
  \label{lem:covering}
  Every integer $n \ge 10^c$ starts with some element of the set
  $S_c := \{a \in \ZZ: 10^c \le a < 10^{c+1}\}$.
\end{lemma}
\begin{proof}
  Let $d = \lfloor \frac{\log n }{\log 10}\rfloor$,
  then $d \ge c$ and $10^d \le n < 10^{d+1}$. Let
  $a = \lfloor \frac{n}{10^{d-c}} \rfloor$. Then
  $10^c \le a < 10^{c+1}$ and
  $a 10^{d-c} \le n < (a+1) 10^{d-c}$. Thus $n$ starts with $a \in S_c$.
\end{proof}
\begin{proposition}
  \label{prop:expansion}
  Let $c \ge 0$ be an integer. Then
  \begin{equation}
    \label{eq:decomposition}
    \sum_{p \text{ prime}} \frac{1}{V(p)} =
    \sum_{p < 10^c \text{ prime}} \frac{1}{V(p)} +
    \sum_{10^c \le a < 10^{c+1}} \sum_{p > 10^c \text{ prime}} \frac{\chi_a(p)}{V(p)}.
  \end{equation}
\end{proposition}
\begin{proposition}
  \label{prop:bounds}
  Let $a > 0, n > 0$ be integers. If the base 10 expansion of
  $n$ starts with $a$
  then
  \begin{equation}
    \label{eq:leading:digits}
    \frac{V(a)}{(a+1)^{\frac{\log 11}{\log 10}}}
      < \frac{V(n)}{n^{\frac{\log 11}{\log 10}}} < 
      \frac{V(a) + 9/10}{a^{\frac{\log 11}{\log 10}}}
  \end{equation}
\end{proposition}
\begin{proof}
  If there is a $c \ge 0 $ such that \eqref{eq:bracket} holds, there is an integer $0 \le d < 10^b$
  such that $n = a 10^c + d$. Thus $V(n) = V(a) 11^c + V(d)$ by
  \lemref{recursion}. However $0 \le V(d) \le \sum_{j=0}^{c-1} 9 \cdot
  11^j < (9/10) 11^b$.
  Thus
  \begin{equation}
    \label{eq:vbracket}
    V(a) 11^c \le V(n) < (V(a) + 9/10)11^c.
  \end{equation}
  The equation \eqref{eq:leading:digits} follows from the inequality
  \eqref{eq:starts} by rearranging:
  \begin{displaymath}
    \frac{\log n}{\log 10} - \frac{\log(a+1)}{\log 10} < c
    \le \frac{\log n}{\log 10} - \frac{\log a}{\log 10},
  \end{displaymath}
  and substituting these bounds for $c$ into \eqref{eq:vbracket}.
\end{proof}
If we take $c=0$ and set $\gamma = \min_{a=1}^9
\frac{V(a)}{(a+1)^{\frac{\log 11}{\log 10}}} > 0$, then $V(n) > \gamma n^{\log 11/\log 10}$.
Thus the series $\sum_n \frac{1}{V(n)}$ is bounded
term by term by the series $\sum_n \frac{1}{\gamma n^{\log 11/\log
    10}}$ which is absolutely convergent.
\begin{definition}
  The \emph{prime zeta function}
  \begin{displaymath}
    P(s) := \sum_{p \text{ prime}} \frac{1}{p^s},
  \end{displaymath}
  where $s \in \CC, \Re s > 1$.
\end{definition}
\begin{lemma}
  \label{lem:interval}
  Let $a$ be a positive integer, $b = \lfloor \log a/\log 10\rfloor$,
  and $F(x) := x - \lfloor x \rfloor$, the fractional part of $x$.
  Define the interval $I_a = [F(\log a/\log 10), F(\log (a+1)/\log
  10)$ if $a < 10^b-1$. If $a=10^b-1$, then $I_a = [F(\log a/\log
  10), 1)$.
  A positive integer $n\ge 10^b$ starts with $a$ if and only if
  $F(\log n\log 10) \in I_a$.
\end{lemma}
\begin{proof}
  By definition, $n$ starts with $a$ if $a 10^c \le n < (a+1)10^c$,
  for some integer $c$. By definition,
  $F(\log a/\log 10) = \log a/\log 10 - b$, where $b = \lfloor \log
  a/\log 10 \rfloor$. Thus
  \begin{displaymath}
    \log a/\log 10 -b \le \log n/\log 10 - c -b < \log(a+1)/\log 10 - b.
  \end{displaymath}
\end{proof}
\begin{lemma}
  \label{lem:upper:bound}
  Let $\Phi(x)$ be a integrable function supported in $[-1,1]$ such
  that $\Phi(x) \ge 0$, and $\int_{-1}^1 \Phi(x) = 1$.  For
  $\frac{1}{4} \ge \varepsilon > 0$, set
  $\Phi_\varepsilon(x) := \varepsilon^{-1} \Phi \left(
    \frac{x}{\varepsilon} \right)$.  Given $a < b$, and $\delta$ Let
  $f_\delta(x)=1$ if $a + \delta \le x \le b - \delta$, and and 0
  otherwise. Then
  $0 \le \Phi_\varepsilon \star f_\delta(x) \le f_0(x)$ for all $x$,
  if $\delta \ge 0$ and
  $\Phi_\varepsilon \star f_\delta(x) \ge f_0(x)$is $\ge f_0(x)$ if $\delta < 0$.
\end{lemma}
\begin{proof}
  Define $\Psi(x) = \int_{\max(-1, x)}^{\min(1,x)} \Phi(t) dt$. Then
  $\Psi(x) = 0$ if $x \le -1$, $\Psi(x) = 1$ if $x \ge 1$ and $\Psi(x)$
  is non-decreasing.
  We have
  \begin{displaymath}
    \begin{aligned}
      \Phi_\varepsilon \star f_\delta(x) & = \varepsilon^{-1}\int_{a +
      \delta}^{b - \delta}
      \Phi((x - t)/\varepsilon) dt \\
      & = \Psi((x-a - \delta)/\varepsilon) -
      \Psi((x-b+\delta)/\varepsilon) \le 1.
    \end{aligned}
  \end{displaymath}
If $(x-b + \delta)/\varepsilon \ge 1$ or $(x-a-\delta)/\varepsilon \le
-1$ the right hand side is 0. Those conditions are $x \ge b + \varepsilon
- \delta$ or $x \le a - \varepsilon + \delta$. If $\delta = \epsilon$
this then shows that $0 \le \Phi_\varepsilon \star f_\varepsilon(x) \le f_0(x)$.
On the other hand, if $x \ge a + \varepsilon + \delta$, and $x \le b -
\delta - \varepsilon$ then the right hand side is 1.
If $\delta = - \varepsilon$, this shows that $\Phi_\varepsilon \star
f_{-\varepsilon}(x) \ge f_0(x)$.
On the other hand, we have $||f_\varepsilon - f_{-\varepsilon}||_1 = 4
\varepsilon$. Thus $||\Phi_\varepsilon \star f_\varepsilon - 
\Phi_\varepsilon \star f_{-\varepsilon}||_\infty \le 4 \varepsilon
||\Phi_\varepsilon||_\infty \le 4 \varepsilon$.
\end{proof}
If we pick a function $\Phi$ as above which is $\mathcal{C}^\infty$
and has all derivatives at $1$ and $-1$ equal to 0, then convolving it
with the indicator function of an interval will produce a ``nice''
function 
\appendix

\section{Some non-context free languages}
\label{sec:nonCFL}

Here we discuss the following language: Let $b > 1$ be an integer, and
$0 < \lambda < 1$. The
alphabet is the set $A = \{0,1,\dots, b-1\}$, and $c \in A$. The language $L$ will be all
non-empty words which
\begin{enumerate}
\item Do not start with 0.
\item If $\sigma \in L$, the number of occurences of $c$ of in
  $\sigma$ is $\le \lambda \ell(\sigma)$, where $\ell(\sigma)$ denotes
  the length of $\sigma$.
\end{enumerate}
\begin{proposition}
  The language $L$ is not context-free.
\end{proposition}
\begin{proof}
  If $L$ where context free, by the pumping lemma, there is a bound,
  $B$ and words
  $\alpha_i, \in A^*$, for $i=0, \dots, 5$, where
  $\ell(\alpha_2 \alpha_3 \alpha_4) \le B$, and
  $\ell(\alpha_2 \alpha_4) > 0$,
  so that
  $\alpha_1 \alpha_2^k \alpha_3 \alpha_4^k \alpha_5 \in L$ for every integer
  $k \ge 1$.

  Denote by $\mu(\sigma)$ the number of occurrences of $c$ in
  $\sigma$. Since $\prod_{i=1}^5 \alpha_i \in L$,
  we have
  \begin{displaymath}
    \mu(\alpha_1) + \mu(\alpha_3) + \mu(\alpha_5) + k (\mu(\alpha_2) +
    \mu(\alpha_4))
    \le \lambda (\ell(\alpha_1) + \ell(\alpha_3) + \ell(\alpha_5)
    + k(\ell(\alpha_2) + \ell(\alpha_4)).
  \end{displaymath}
  Dividing both sides by $k$ and letting $k \rightarrow \infty$ we
  then have
  \begin{displaymath}
    \mu(\alpha_2) + \mu(\alpha_4) \le \ell(\alpha_2) + \ell(\alpha_4).
  \end{displaymath}
  
\end{proof}

\nocite{johnson2015saddle}
\nocite{baillie2024summing}
\nocite{schmelzer2008summing}
\nocite{burnol2024moments}
\nocite{maynard2019primes}
\nocite{baillie1979sums}
\bibliography{series,digits}
\bibliographystyle{plain}
\end{document}
